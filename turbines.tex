\documentclass[12pt,a4paper,twoside]{extreport}
\usepackage[utf8]{inputenc}
\usepackage[LGR,T1]{fontenc}
\usepackage[greek,english]{babel}
\usepackage{alphabeta}
\usepackage{amsmath}
\usepackage{tocloft}
\usepackage{graphicx}
\usepackage{multicol}
\usepackage{listings}
\usepackage{amssymb}
\usepackage{subcaption}
\usepackage{caption}
\usepackage{pgfplots}
\usepackage{filecontents}
%\usepackage{titling}
\usepackage{fancyhdr}
\usepackage{etoolbox}
\usepackage[nottoc,notlof,notlot]{tocbibind}
\usepackage{lastpage}
\usepackage{blindtext}
\usepackage{graphicx}
\usepackage{array}
\usepackage{tabularx}
\usepackage{stackengine,lipsum}
\setstackEOL{\\}
\usepackage{floatrow}
\floatsetup[table]{capposition=top}
\usepackage{tikz}
\usetikzlibrary{decorations.markings}
\usetikzlibrary{datavisualization}
\usetikzlibrary{shapes.geometric, arrows, shadows,positioning}
\usepackage[left=2.5cm,right=2.5cm,top=2.5cm,bottom=2.5cm]{geometry}
\usepackage{indentfirst}
\renewcommand{\cftchapleader}{\cftdotfill{\cftdotsep}}
\renewcommand{\cftsecleader}{\cftdotfill{\cftdotsep}}
\addto\captionsenglish{%
  \renewcommand{\contentsname}{}%
   \renewcommand{\chaptername}{}%
   \renewcommand{\thechapter}{}%
    \renewcommand{\figurename}{}%
    \renewcommand{\bibname}{Βιβλιογραφία}
     \renewcommand{\tablename}{Πίνακας}
}

%define header
\patchcmd{\chapter}{\thispagestyle{plain}}{\thispagestyle{fancy}}{}{}

\makeatletter
\newcommand{\rightorleftmark}{%
  \begingroup\protected@edef\x{\rightmark}%
  \ifx\x\@empty
    \endgroup\leftmark
  \else
    \endgroup\rightmark
  \fi}
\makeatother

\pagestyle{fancy}
\fancyhf{}
\fancyhead{}
\renewcommand{\headrulewidth}{0pt}
\fancyhead[RE]{\slshape \fontsize{10}{10} \leftmark}
\fancyhead[RO]{\slshape \fontsize{10}{10} \rightorleftmark}
\fancyhead[L]{\includegraphics[width=1.5cm]{ntualogo.png} 
\Longstack[l]{\textbf{\fontsize{10}{10} Εθνικό Μετσόβιο Πολυτεχνείο} \\  \fontsize{10}{10} Σχολή Μηχανολόγων Μηχανικών
}}
\setlength{\headheight}{45pt}
\cfoot{Μάιους}
\rfoot{\thepage \ of \pageref{LastPage}}


\author{Κοράκης Βασίλειος, mc17110}
\title{\underline{\textbf{Θέματα Στροβιλομηχανών}}  \\
 \large  \vspace{0.5cm}  Θερμικές στροβιλομηχανές, $5^o$ εξάμηνο \vfill
 }
\date{\vfill Αθήνα,Μάιος 2020}




\begin{document}
\maketitle
\chapter*{Δεδομένα εισόδου}
Oι παράμετροι ονόματος λαμβάνονται: με $ON=2, EΠ=10, ΠΑ=2$, ως $\\ Δ1=0.9087, \\Δ2=0.9783,\\Δ3=0.9089$
\newpage
\chapter{Θέμα Α}
\section*{$A1$)}
Με τις παραδοχές της σταθερής αξονικής/περιφερειακής ταχύτητας και της επαναληπτικής βαθμίδας στροβίλου, ισχύει:
\begin{center}
    \begin{align*}
    tan(Δα) & = \frac{-ΦΨ}{Φ^2 +(1-r)^2 - \frac{Ψ^2}{4} } \\
    \text{ή} \qquad    \frac{Ψ^2}{4}-Φ^2-\frac{ΦΨ}{tan(Δα)} & = (1-r)^2 
    \end{align*}
\end{center}
θέτοντας $Ψ = αψ + βφ, Φ=γψ + δφ$ η εξίσωση γίνεται:
\begin{center}
    $\displaystyle (1-r)^2 = \frac{-4βδ + β^2 tan(Δα)-4δ^2tan(Δα)}{4tan(Δα)}Φ^2 + \frac{-2αδ-2βγ+αβtan(Δα)-4δγtan(Δα)}{2tan(Δα)}ΦΨ+ \frac{α^2tan(Δα)-4αγ-4γ^2tan(Δα)}{4tan(Δα)}Ψ^2$
\end{center}
οπότε για:
\begin{equation*}
\left\{ \begin{array}
  {lr} α=δ=1 \\  γ=0
\end{array} \right.
\end{equation*}
και μηδενίζοντας των όρο που περιέχει $ΦΨ$, προκύπτει ότι $β=\frac{2}{tan(Δα)}$. Άρα,
\begin{equation*}
\left \{ \begin{array}
  {lr} Φ=φ \\  Ψ=ψ(φ)+φ\frac{2}{tan(Δα)}
\end{array}\right.
\end{equation*}
με $ψ=\pm \frac{2\sqrt{φ^2+tan^2(Δα)(1-2r+r^2+φ^2}}{tan(Δα)}$
\newline μέσω δοκιμής επιλέγεται η \textbf{αρνητική} λύση για γωνίες κάτω τον 90 (απόλυτα) και η \textbf{θετική} λύση για γωνίες πάνω. Επίσης, τίθενται οι  γωνίες $Δα=α1-α2=α1-α3$:
\newline $-65 \quad -80\quad-95\quad-110\quad-125\quad\text{(βήμα $15^o$)}$

Έτσι, για r=0.656, γίνεται το διάγραμμα:

%%%%%%%%%%%%%%%%%%%%%%%%%%%%%%%%%%%%%%%%%%%%%%%%%%%%%%%%%%%%%%%%%%%%%%%%%%%
\newpage
\section*{$A2$)}
Με την ίδια λογική:
\begin{center}
    \begin{align*}
    tan(Δβ) & = \frac{ΦΨ}{Φ^2 +r^2 - \frac{Ψ^2}{4} } \\
    \text{ή} \qquad    \frac{Ψ^2}{4}-Φ^2+\frac{ΦΨ}{tan(Δβ)} & = r^2 
    \end{align*}
\end{center}
και πάλι θέτοντας $Ψ = αψ + βφ, Φ=γψ + δφ$ η εξίσωση γίνεται:
\begin{center}
    $\displaystyle r^2 = \frac{-4βδ + β^2 tan(Δβ)-4δ^2tan(Δβ)}{4tan(Δβ)}Φ^2 + \frac{+2αδ+2βγ+αβtan(Δβ)-4δγtan(Δβ)}{2tan(Δβ)}ΦΨ+ \frac{α^2tan(Δβ)+4αγ-4γ^2tan(Δβ)}{4tan(Δβ)}Ψ^2$
\end{center}
οπότε για:
\begin{equation*}
\left\{ \begin{array}
  {lr} α=δ=1 \\  γ=0
\end{array} \right.
\end{equation*}
και μηδενίζοντας των όρο που περιέχει $ΦΨ$, προκύπτει ότι $β=\frac{-2}{tan(Δβ)}$. Άρα,
\begin{equation*}
\left \{ \begin{array}
  {lr} Φ=φ \\  Ψ=ψ(φ)+φ\frac{-2}{tan(Δβ)}  
\end{array}\right.
\end{equation*}
με $ψ=\pm \frac{2\sqrt{r^2tan^2(Δβ)+(1+tan^2(Δβ)Φ^2}}{tan(Δβ)}$
\newline και πάλι μέσω δοκιμών επιλέγεται η \textbf{θετική} λύση για γωνίες κάτω τον 90, και η \textbf{αρνητική} λύση για πάνω. Οι γωνίες τίθενται ως, $Δβ=β2-β3$:
\newline $50 \quad 65 \quad 80 \quad 95 \quad 110 \quad\ \text{(βήμα $15^o$)}$
\newline Έτσι, γίνεται το διάγραμμα:
%%%%%%%%%%%%%%%%%%%%%%%%%%%%%%%%%%%%%%%%%%%%%%%%%%%%%%%%%%%%%%%
\newpage
\section*{$A3$)}
Για την γωνία εξόδου του ακροφυσίου, $α2$, ισχύει:
\begin{center}
    \begin{align*}
        (1-r+Ψ)\frac{1}{Φ} & = tan(α2) \\
        \text{ή} \qquad Ψ & = 2(Φtan(α2)-1+r)
            \end{align*}
\end{center}
με τις γωνίες να επιλέγονται ως:
\newline $10 \quad 23\quad36\quad49\quad62\quad\text{(βήμα $13^o$)}$

%%%%%%%%%%%%%%%%%%%%%%%%%%%%%%%%%%%%%%%%%%%%%%%%%%%%%%%%%%%%%%%%%%%
\newpage
\section*{$A4$)}
Οι μεσαίες καμπύλες τέμνονται στα σημεία:
\newline
\begin{tabular}{c c}
\left \{ \begin{array}
  {lr} α1-α2  = -95 \\ β2-β3  = 80
\end{array}\right.
\hspace{1cm}
&
κ'
\hspace{1cm}

\left \{ \begin{array}
  {lr} Φ = 0.7 \\  Ψ = 1.688
\end{array}\right.
\\ 
 &
\end{tabular}

 
\par Οπότε βρίσκονται τα μεγέθη του τριγώνου ταχυτήτων:
\newline
\begin{equation*}
\left\{ \begin{array}
  {lr} α1=α3=Arctan(\frac{1}{Φ}(1-r-\frac{Ψ}{2}))=-35.54^o \\
  α2 = Arctan(\frac{1}{Φ}(1-r+\frac{Ψ}{2}))=59.49^o \\
  β2 = Arctan(\frac{1}{Φ}(-r+\frac{Ψ}{2}))=15.03^o \\
  β3 = Arctan(\frac{1}{Φ}(-r-\frac{Ψ}{2}))=-64.98^o
\end{array} \right.
\end{equation*}
\par Τα πτερύγια έχουν μορφή παραβολής που ξεκινάει από το (0,0), οι γωνίες <<εισόδου-εξόδου>> ή <<αριστερά-δεξιά>> όπως θα συμβολιστούν παρακάτω, είναι καθορισμένες, και είναι είτε απόλυτες γωνίες, α, για το ακροφύσιο, είτε σχετικές γωνίες-γωνίες μετάλου, β, για τον ρότορα. Η εφαπτομένη της καμπύλης που πρέπει να ισούται με τις γωνίες αυτές:
\begin{tabular}{c c}
\left \{ \begin{array}
  {lr} y = ax^2 +bx \\ y' = 2ax+b
\end{array}\right.

\hspace{1cm}
&
κ'
\hspace{1cm}

\left \{ \begin{array}
  {lr} tan(α_L) = y'(0)=b \\  tan(α_R) = y'(L)=2aL+b
\end{array}\right.

\\ 
 &

\end{tabular}
\newline δίνει τελικά την εξίσωση πτερυγίου $\displaystyle{y = \frac{tan(α_R)-tan(α_L)}{2L}x^2+tan(α_L)x}$ .
\par Εφόσον η χορδή (διάνυσμα r στο παρακάτω σχήμα) έχει μήκος 1, πρέπει $|\Vec{r}|=1$ ή $\sqrt{L^2+y^2(L)}=1$ και τελικά το L υπολογίζεται:
\newline
$\displaystyle L = \frac{1}{\sqrt{1+(\frac{\tan(α_R)-tan(α_L)}{2}+tan(α_L))^2}}$
\begin{figure}[h]
\centering
\begin{tikzpicture}[set mark/.style args={#1 at #2}{postaction={decorate,
decoration={markings,mark=at position #2 with #1}}}]
\draw [<->,thick] (0,2) node (yaxis) [above] {$y$}
        |- (3,0) node (xaxis) [right] {$x$};
\coordinate (p1) at (0,0);
\coordinate (p2) at (2,2);
\coordinate (p3) at (1.88,1.93);
\draw[gray,thick] (p1) edge[bend left=20,distance=1cm] (p2);
\filldraw[black] (p3) circle (1pt);
\draw [->,very thick] (p1) to node [at end,below]{$\scriptstyle{\Vec{r}=(L,y(L))}$} (p3) ;
\end{tikzpicture}
\caption*{\small{βοηθητικό σχήμα ορισμού χορδής}}
\end{figure}
\newline \underline{Πτερύγιο Στάτορα}:
\newline $α_L=α1, \qquad α_R=α2$
\newline $L_s=0.8975, \qquad \boxed{y_s=1.3434x^2-0.7143x}$
\newline \underline{Πτερύγιο ρότορα}:
\newline $α_L=β2, \qquad α_R=β3$
\newline $L_R=0.7297, \qquad \boxed{y_R=-1.6524x^2+0.2686x}$
\newline
\vspace{1cm}
s
\chapter{}
\chapter{}



\newpage
\end{document}